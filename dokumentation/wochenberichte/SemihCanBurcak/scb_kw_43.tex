\subsubsection{KW43: 25.10.2021 bis 31.10.2021}
\begin{quote}
	\subsubsection*{Arbeit in der Schule}
	**HERBSTFERIEN** 
	
	\subsubsection*{Arbeit außerhalb der Schule}
	Ich habe das Tutorial-Spiel\footcite{Karting Microgame Unity3D} entsprechend zu dem Tutorial\footcite{Unity Photon Tutorial} fertig umprogrammiert. Ich musste diese Woche nur noch die seperate Bewegung und Synchronisation der Spieler programmieren. Aber da der Code vom Spiel nicht von mir stammt und auch sehr kompliziert war, musste ich viel mit Fehlern kämpfen. Doch ein Fehler konnte ich nicht lösen. Der Bewegung-Code hatte noch die Photon-Bibliothek nicht und es erkannte sie auch nicht. Ich musste zuerst die Photon-Bibliothek Referenzen einfügen, was leider nicht nur mit einem Rechtsklick auf Referenzen zu machen war, obwohl es so machbar sein sollte, es hat mich sehr viel Zeit gekostet aber zumindest hat Visual Studio die Photon-Bibliothek erkennen können. Doch dann kam der nächste Fehler, der Code wurde nicht kompiliert von Unity. Ich habe den Fehler mit paar Tricks, wie z.B. sehr viele Leerzeichen einfügen, lösen können aber auch nachdem Unity mein Code kompiliert hat, bekam ich immer noch den Fehler, dass die Photon-Bibliothek nicht gebunden war und deshalb manche Sachen im Code nicht erkennt werden können.     
	
	\begin{itemize}
		\item Spiel fertig umprogrammiert - 1h
		\item Fehlersuche - 5h
	\end{itemize}
	
	\subsubsection*{Was ist geplant für die Nächste Woche}
	Eine Datenbank erstellen mit MySQL und es mit Unity verbinden.
\end{quote}