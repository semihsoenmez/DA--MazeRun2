\subsubsection{KW42: 18.10.2021 bis 24.10.2021}
\begin{quote}
	\subsubsection*{Arbeit in der Schule}
	In der Schule habe ich über Online-Multiplayer-Spielentwicklung (OMSE) in Unity recherchiert. Es gibt eine große Auswahl von Services bzw. Bibliotheken, die man wählen kann für OMSE, aber ich habe Photon gewählt, weil Sie gratis einen cloudbasierten Webserver anbieten und das Service ist auf C\# basiert.
	
	Für Mehreres hatte ich leider keine Zeit mehr, da wir nach der zweiten Stunde für Maturafotos fort müssten.  
	
	\subsubsection*{Arbeit außerhalb der Schule}
	Ich habe ein Tutorial-Spiel\footcite{Karting Microgame Unity3D} heruntergeladen und habe angefangen mithilfe Photon, und mit einem entsprechenden Tutorial\footcite{Unity Photon Tutorial} dazu, das Spiel in ein Online-Multiplayer-Spiel zu umprogrammieren. Da es mein erstes Mal war mit Photon, musste ich zuerst mal die Lernkurve überwinden. Ich bin noch nicht fertig geworden mit dem Umprogrammieren. Ich habe geschafft das Spiel mit dem Server zu verbinden und Räume zu erstellen, ich konnte mit zwei Builds in den gleichen Raum beitreten.
	
	Ich habe mich diese Woche mit den Menü-Designs nicht beschäftigt, obwohl es geplant war. Weil, ich denke, dass es momentan nicht sehr nötig ist bzw. eine hohe Priorität hat.   
	
	\begin{itemize}
		\item Recherche über Photon - 1h
		\item Tutorial-Spiel auf Online-Spiel umprogrammiert - 3h
	\end{itemize}
	
	\subsubsection*{Was ist geplant für die Nächste Woche}
	Das Tutorial-Spiel fertig umprogrammieren.
\end{quote}