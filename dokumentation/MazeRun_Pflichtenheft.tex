

\emph{\textbf{Diplomarbeit "MazeRun" 5AHEL 2021-22}}

\hypertarget{lastenheft--pflichtenheft}{%
\section{Lastenheft / Pflichtenheft}\label{lastenheft--pflichtenheft}}

\hypertarget{kurzbeschreibung}{%
\subsection{Kurzbeschreibung}\label{kurzbeschreibung}}
\begin{quote}
	Das Spiel ist in der 3D-Umgebung in Unity zu realisieren. Die Highscores
	werden auf einem zentralen Server gespeichert. Die Eingaben vom
	Kontroller sollen von der Megacard/ESP ausgelesen und mittels USB/WLAN
	evtl. Bluetooth an das Spiel weitergeitet werden. Anschließend sollte
	sich die Spielfigur entsprechend bewegen. Der aus einem eigenen Akku und
	Gehäuse bestehender Kontroller soll auch als Feedback beim Spielen
	vibrieren. Die Diplomarbeit soll auch ein How-To für Erstellung eines
	Unity-Spiels/Megacard \& ESP Kontrollers sein.
\end{quote}


\hypertarget{aufgaben-einzelner-projektmitglieder}{%
\subsection{Aufgaben einzelner
Projektmitglieder}\label{aufgaben-einzelner-projektmitglieder}}

\begin{longtable}{l l}

\textbf{Name} & \textbf{Individuelle Themenstellung}\tabularnewline

\endhead
Semih Can Burcak (Hauptverantwortlich) & \makecell[tl]{Projektmanagement /
	Spielsoftwareentwicklung \\ / Spieldesign / Onlinespielentwicklung  } \\
Semih Sönmez & \makecell[tl]{Spielsoftwareentwicklung / Spieldesign} \\
Manuel Rath & \makecell[tl]{Schaltungsdesign / Analoglayout \\/ Microcontroller
Programmierung} \\
Mateo Herceg & \makecell[tl]{Schaltungsdesign / Analoglayout \\/ Microcontroller
Programmierung} \\

\end{longtable}

\hypertarget{projektziele-nach-smart}{%
\subsection{Projektziele nach SMART}\label{projektziele-nach-smart}}
\begin{quote}

\hypertarget{spezifisch}{%
\subsubsection{Spezifisch}\label{spezifisch}}

Das Spiel soll professionell aussehen und ohne Probleme online von jeden
gespielt werden können. Die Spielstände der Spieler sollen einwandfrei
auf einem Server gespeichert werden. Die selbstgemachten Kontroller
sollen ebenfalls einwandfrei funktionieren. Die Verbindung zwischen dem
Spiel (C\#) und Kontroller (C) soll automatisch sein bzw. die Kontroller
sollen Plug \& Play Kontroller sein.

\hypertarget{messbar}{
\subsubsection{Messbar}\label{messbar}}

Jede neue Funktion des Spiels soll ausführlich getestet werden. Die
Verbindung des Kontrollers zum Spiel wird im Spiel getestet, indem wir
schauen ob die Tasten und die Joysticks das machen, was wir wollen.

Nachdem das Spiel und die Kontroller fertig entwickelt wurden, wird
alles nochmals getestet um evtl. Bugs zu finden und zu schauen ob alles
wie geplant funktioniert.

\subsubsection{Realistisch}\label{realistisch}

Die Programmiersprachen C\# und C wurden im Unterricht schon
durchgenommen. Die Kommunikation zwischen PC und µC wurde noch nicht
gemacht, aber mit der richtigen Hilfe bzw. Professor sollte es
realisierbar sein. Das Projekt wird Schritt für Schritt entwickelt und
ist daher realistisch.

\hypertarget{terminiert}{%
\subsubsection{Terminiert}\label{terminiert}}


\begin{longtable}[]{@{}ll@{}}

\emph{Datum} & \emph{Meilenstein}\tabularnewline

\endhead

\textbf{Anfang Oktober (04.10.2021)} & Auswahl und Layout der Hardware,
Anfangen das Spiel zu programmieren\tabularnewline
\textbf{Anfang November (01.11.2021)} & Die Umgebung/Charaktere des
Spiels sollen fertig sein,\tabularnewline
\textbf{Mitte bis Ende November (30.11.2013)} & Entwicklungsphase des
Spiels fertig,\tabularnewline
\textbf{Dezember und Jänner} & Online-Funktion des Spiels
programmieren\tabularnewline
\textbf{Ende 1. Halbjahr (14.02.2022)} & Fertigstellung des
Projekts\tabularnewline

\caption{Meilensteine}

\end{longtable}

Das Projekt soll in diesem Halbjahr (Wintersemester SJ 2021-22)
fertiggestellt werden.

\end{quote}

\hypertarget{produkteinsatz}{%
\subsection{Produkteinsatz}\label{produkteinsatz}}
\begin{quote}
Das Spiel sollte mit einem Computer gestartet werden, der Zugang auf ein
Internet verfügt. Die Kontroller sollten entweder über USB, Bluetooth
oder WLAN mit dem PC verbunden sein. Bei kabellosem Kontroller soll
sicher gestellt werden, dass die Batterie noch Energie hat.
\end{quote}
\hypertarget{funktionale-anforderungen}{%
\subsection{Funktionale Anforderungen}\label{funktionale-anforderungen}}
\begin{quote}
Durch den Kontroller sollten folgende Bewegungen durchgeführt werden
können:

\begin{itemize}
\item
  Grundlegende Bewegungen (vorne, hinten, rechts, links,)
\item
  Springen des Spielers
\item
  Bewegung der Kamera
\end{itemize}
\end{quote}
\hypertarget{projektphasen-und-meilensteine}{%
\subsection{Projektphasen und
Meilensteine}\label{projektphasen-und-meilensteine}}
 
\hypertarget{spiel}{%
\subsubsection{Spiel}\label{spiel}}
\begin{quote}
Zuerst sollen die Grundlegenden Bewegungen vom Charakter programmiert
werden. Anschließend kann das Multiplayer Spielmodus programmiert
werden. Das Know-How soll bis das Spiel bereit ist (Bewegungen, Umgebung
etc.), um es online zu spielen, erschafft werden.
\end{quote}
\hypertarget{kontroller}{%
\subsubsection{Kontroller}\label{kontroller}}
\begin{quote}
Zuerst werden Layout und Schaltplan erstellt und daraus wird die
Leiterplatte entstehen. In der zwischen Zeit werden die einzelnen
Bauteile bestellt und Angesteuert. Nach der Fertigung der Hardware
starten wir mit einzelnen Testprogrammen, die wir später für die fertige
Software brauchen. Zu guter Letzt wird ein passendes Gehäuse mit einem
Akku konstruiert.
\end{quote}
\hypertarget{arbeitspakete-aufgelistet}{%
\subsection{Arbeitspakete
(aufgelistet)}\label{arbeitspakete-aufgelistet}}

\hypertarget{semih-can-burcak}{%
\subsubsection{Semih Can Burcak}\label{semih-can-burcak}}

\begin{itemize}
\item
  Benutzeroberfläche des Spiels

  \begin{itemize}
  \item
    Username Eingabe - must have
  \item
    Wahlmenü (Einzelspieler oder Mehrspieler, Level usw.) - must have
  \item
    Es soll aus HTL Rankweil Farben bestehen (Blau, Grau, Schwarz)
  \item
    Benutzerfreundlich gestalten - must have
  \end{itemize}
\item
  Audio

  \begin{itemize}
  \item
    Copyrightfreie Audios verwenden - must have 
  \item
    Die Musik und die Sounds sollen dem Spiel mehr energetisch machen -
    must have
  \item
    Lautstärke gut einstellen (nicht zu laut oder leise)
  \end{itemize}
\item
  Multiplayer - must have

  \begin{itemize}
  \item
    Webserver einrichten 
  \item
    Datenbank erstellen
  \item
    Datenbank und Webserver mit Unity verbinden
  \item
    Schauen, dass jeder Spieler ein Charakter zugeteilt kriegt
  \end{itemize}
\item
  Story

  \begin{itemize}
  \item
    Es soll nicht langweilig sein und Interesse wecken - must have
  \item
    Zuerst auf dem Papier das Konzept bzw. Story festigen und erst
    programmieren
  \end{itemize}
\item
  Projektmanagement - must have

  \begin{itemize}
  \item
    Das Team soll immer organisiert bleiben
  \item
    Jeder soll seine Aufgaben rechtzeitig erledigen, damit kein
    Zeitdruck entsteht
  \item
    Ordentlich Dokumentieren
  \end{itemize}
\end{itemize}

\hypertarget{semih-suxf6nmez}{%
\subsubsection{Semih Sönmez}\label{semih-suxf6nmez}}

\begin{itemize}
\item
  Entwicklung des Hauptcharakters

  \begin{itemize}
  \item
    Grundlegende Bewegungen (vorne, hinten, rechts, links) -must have
  \item
    Springfunktion um über Hindernissen zu springen -must have
  \item
    Schussfunktion um Gegner zu eliminieren
  \end{itemize}
\item
  Entwicklung der Gegner

  \begin{itemize}
  \item
    Die Gegner sollen zum Hauptcharakter rennen -must have
  \item
    Bei einem Kontakt, soll das Leben vom Hauptcharakter weniger werden
    -must have
  \end{itemize}
\item
  Umgebung

  \begin{itemize}
  \item
    Die Charaktere sollen die Umgebung nicht verlassen dürfen -must have
  \item
    Die Umgebung soll gut beleuchtet sein -must have
  \end{itemize}
\item
  Missionen \& Levels

  \begin{itemize}
  \item
    5 verschiedene Levels -must have
  \item
    Missionen sollen interessant und bewältigbar sein -must have
  \end{itemize}
\item
  Kamera

  \begin{itemize}
  \item
    Bei Einzelspielermodus soll die Kamera den Hauptcharakter folgen
    -must have
  \item
    Bei Multiplayermodus soll die Kamera für beide Spieler angepasst
    werden -must have
  \end{itemize}
\end{itemize}

\hypertarget{manuel-rath}{%
\subsubsection{Manuel Rath }\label{manuel-rath}}

\begin{itemize}
\item
  Leiterplatte - must have

  \begin{itemize}
  \item
    Schaltplan mit geeigneten Bauteilen \& Stromversorgungsschaltung
    erstellen
  \item
    Layout möglichst kompackt
  \end{itemize}
\item
  Gehäuse - must have

  \begin{itemize}
  \item
    handliches und nicht zu großes Gehäuse 
  \item
    mit 3D Drucker erstellen 
  \end{itemize}
\item
  Verbindung von Kontroller zum PC

  \begin{itemize}
  \item
    Daten zuerst über UART an den PC senden - must have
  \item
    Daten mit Bluetooth Modul an den PC senden
  \end{itemize}
\item
  Komponenten Ansteuern mit der Megacard - must have

  \begin{itemize}
  \item
    Joystick ansteuern und daten auslesen
  \item
    Vibrationsmotor ansteuern 
  \item
    Tasten auslesen 
  \item
    Programme in einem Spiel testen 
  \end{itemize}
\end{itemize}

\hypertarget{mateo-herceg}{%
\subsubsection{Mateo Herceg}\label{mateo-herceg}}

\begin{itemize}
\item
  Leiterplatte - must have

  \begin{itemize}
  \item
    Layout und Schaltplan mit diversen Bauteilen (+ESP Modul) entwickeln
    \& Stromversorgungsschaltung hinzu entwickeln
  \item
    Strukturiertes Layout + übersichtlicher Schaltplan 
  \end{itemize}
\item
  Verbindung Kontroller/Computer

  \begin{itemize}
  \item
    Daten mittels ESP Modul über WLAN an den PC senden - must have
  \item
    Daten mittels ESP Modul über Bluetooth an den PC senden
  \end{itemize}
\item
  Ansteuerung - must have

  \begin{itemize}
  \item
    Knöpfe/Tasten ansteuern und Signale/Daten auslesen mittels ESP Modul
  \item
    Joystick ansteuern und Signale/Daten auslesen mittels ESP Modul
  \item
    Vibrationsmotor ansteuern mittels ESP Modul
  \item
    Alle angesteuerten Komponenten an PC senden und Testen
  \end{itemize}
\item
  Gehäuse - must have

  \begin{itemize}
  \item
    Gehäuse konstruieren
  \item
    Gehäuse mittels 3D Drucker erstellen
  \item
    einfaches handliches Gehäuse
  \end{itemize}
\end{itemize}


