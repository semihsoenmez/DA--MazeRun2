% Preamble (Angaben gelten für das ganze Dokument)
% Hilfe für Bibtex Einträge: https://www.literatur-generator.de
% ------------------------------------------------
% rote Umrandungen bei Verlinkungen entfernen: hidelinks
% 12pt als Standardschriftgröße
% A4-Dokument
% twoside (zweiseitiges Dokument): nur so werden verschiedene Fußzeilen für gerade bzw. ungerade Seitenzahlen übernommen
% article: Dokumentvorlage
\documentclass[hidelinks,12pt,a4paper,twoside]{article}
% Einbinden von Paketen
% ---------------------
\usepackage{titlesec} 						% ermöglicht die Verwendung der title- und section-Befehle
\usepackage[utf8]{inputenc}					% ermöglicht das Verwenden von Umlauten - linux od osx
\usepackage{makecell}
\usepackage[top=3.5cm,left=3cm,right=2cm,bottom=2.5cm,headsep=0.3in,headheight=1in]{geometry} % Seitenränder
\usepackage{graphicx} 						% ermöglicht die Verwendung des includegraphics-Befehls zur Einbindung von Bildern
\usepackage{fancyhdr} 						% ermöglicht die einfache Erstellung von Kopf- und Fußzeilen
\usepackage{setspace} 						% ermöglicht Veränderung verschiedener Abstände
\usepackage{array} 							% ermöglicht unter anderem das Erstellen von benutzerdefinierten Vorlagen für Tabellenspalten
\usepackage[export]{adjustbox} 				% Umrandung von Bildern
\usepackage[nottoc,numbib]{tocbibind} 		% ermöglicht die Erstellung eines Inhaltsverzeichnisses
\usepackage{tocloft}
\setlength{\cftsubsecnumwidth}{3em} 		% Set distance between number and text of subsection in table of contents
\usepackage{color} 							% ermöglicht das Erstellen von benutzerdefinierten Farben

\usepackage{longtable}						% Unterstützung für mehrseitige Tabellen
\usepackage[ngerman]{babel}					% Deutsche Bezeichnungen 

\usepackage[backend=biber,citestyle=verbose, bibstyle=verbose]{biblatex} %mehrsprachiges Bibliotheks(Literatur-)verzeichnis
\usepackage[babel,german=guillemets]{csquotes} %deutsches Anführungszeichen
\addbibresource{bibliography.bib}

\usepackage{blindtext} 						% Blindtext
\usepackage{url} 							% Formatierung von URLs
\usepackage[bottom,hang,stable]{footmisc} 	% Fußnoten immer am unteren Ende der Seite
\usepackage{caption} 						% ermöglicht das Anpassen von diversen Beschriftungen
\usepackage{subcaption} 					% ermöglicht das Erstellen von Unterbezeichnungen (subfigures)
\usepackage{hyperref} 						% Hyperlinks im Dokument
\usepackage{pdfpages} 						% einbinden von PDF-Dateien
\usepackage{amsmath} 						% grundlegende mathematische Funktionen

\usepackage{listings,chngcntr} 				% kapitelweise Nummerierung für Codeabschnitte
\usepackage{listings,xcolor} 				% ermöglicht das Einbinden von Codesegmenten
\usepackage{listingsutf8} 					% ermöglicht das Einbinden von Codesegmenten

\usepackage{float} 							% Positionierung verschiedener Objekte
\usepackage{footnote} 						% ermöglicht das Anpassen von Fußnoten
\usepackage{enumitem} 						% ermöglicht das Anpassen von Aufzählungen
\usepackage{longtable} 						% erlaubt mehrseitige Tabellen
\usepackage{url} 							% gibt eine schöner formatierte Internetadresse aus


% allgemein gültige Formateinstellungen
% -------------------------------------
\renewcommand{\familydefault}{\sfdefault} 	% set Font-Family to a similar font like Arial
\raggedbottom 								% Abschnitte werden nicht auseinander gezogen, um den restlichen Platz auf einer Seite zu füllen
%\raggedright 								% gesamtes Dokument linksbündig (auch Paragraphen) OHNE Silbentrennung

\usepackage{ragged2e}						% gesamtes Dokument linksbündig (auch Paragraphen) MIT Silbentrennung
\RaggedRight

\onehalfspacing 							% 1.5-facher Abstand zwischen den Zeilen

\renewcommand{\arraystretch}{1.2} 			% Vergrößerung des Abstands zwischen den Tabellenzeilen
\newcolumntype{C}[1]{>{\centering\arraybackslash}p{#1}} % erstellen eines Spaltentyps mit zentriertem Inhalt unter der Angabe einer Spaltenbreite
\newcolumntype{L}[1]{>{\raggedright\arraybackslash}p{#1}} % erstellen eines Spaltentyps mit linksbündigem Inhalt unter der Angabe einer Spaltenbreite

\setlength{\footnotemargin}{0.5cm} 			% Abstand zwischen Fußnotennummer und -text
\setlist{nosep} 							% zusätzliche Abstände bei Aufzählungen entfernen
\setlength{\parskip}{1em} 					% Abstand nach einem Absatz

% Abstände vor und nach Überschriften auf den verschiedenen Ebenen
\titlespacing\section       {0pt}{0pt plus 0pt minus 2pt}{0pt plus 2pt minus 2pt}
\titlespacing\subsection    {0pt}{0pt plus 0pt minus 2pt}{0pt plus 2pt minus 2pt}
\titlespacing\subsubsection {0pt}{0pt plus 0pt minus 2pt}{0pt plus 2pt minus 2pt}

% Formateinstellungen für Codesegmente
% ------------------------------------
\usepackage{caption}
\usepackage{minted} 						% für schöne Darstellung von Code
\usepackage{dingbat}						% für Sonderzeichen wie \carriagereturn

\setminted{
	autogobble,								% rückt Code soweit nach links wie möglich
	stepnumber=2,							% nur jede gerade Zeilennummer wird angezeigt
	fontfamily=tt,
	linenos=true,
	numberblanklines=true,
	numbersep=5pt,
	gobble=0,
	frame=single,
	framerule=0.4pt,
	framesep=2mm,
	funcnamehighlighting=true,
	tabsize=4,
	obeytabs=false,
	breaklines,
	breakafter=">/,
	breakbefore=. ,
	breaksymbolindentright=10pt,
	breaksymbolsepright=0pt
}

\newmintedfile[csharpcode]{csharp}{}
\newmintedfile[xamlcode]{xml}{}
\newmintedfile[bashcode]{bash}{}

\usepackage{etoolbox}
\AtBeginEnvironment{longlisting}{\dontdofcolorbox}
\def\dontdofcolorbox{\renewcommand\fcolorbox[4][]{##4}}		% zum Unterbinden des roten Rahmens im Quellcode um ein $-Zeichen

\newenvironment{longlisting}{\captionsetup{type=listing} \linespread{1.1}}{} % 1.1 facher Zeilenabstand bei listings

% Anlegen von benutzerdefinierten Farben für das Hervorheben bestimmter Teile des Codes
\definecolor{shblue}{rgb}{0.13,0.13,1} % Farbe für Schlüsselwörter
\definecolor{shgreen}{rgb}{0,0.5,0} % Farbe von Kommentaren
\definecolor{shred}{rgb}{0.9,0,0} % Farbe von Zeichenfolgen
\definecolor{lightgray}{gray}{0.95} % Hintergrundfarbe
\definecolor{maroon}{rgb}{0.5,0,0}
\definecolor{bg}{rgb}{0.95,0.99,0.99}

% benutzerdefiniertes Syntax-Highlighting für XAML
\lstdefinelanguage{XAML}
{
	morestring=[s]{"}{"},
	moredelim=[s][\color{black}]{>}{<},
	moredelim=[s][\color{maroon}]{<}{\ },
	moredelim=[s][\color{maroon}]{</}{>},
	moredelim=[l][\color{maroon}]{/>},
	moredelim=[l][\color{maroon}]{>},
	morecomment=[s]{<?}{?>},
	morecomment=[s]{<!--}{-->},
	commentstyle=\color{shgreen},
	stringstyle=\color{shblue},
	identifierstyle=\color{shred}
}

% spezifische Einstellungen für verschiedene Sprachen
\lstdefinestyle{csharp}
{
	language=[Sharp]C,
	commentstyle=\color{shgreen}, % Farbe für Kommentare
	keywordstyle=\color{shblue},
	stringstyle=\color{shred}
}

\lstdefinestyle{xaml}
{
	language=XAML
}

\skip\footins 30pt				% etwas größerer Abstand oberhalb Fußnote

% Anpassung diverser Beschriftungen
\captionsetup[listing] {labelfont=bf,textfont=it,format=hang,justification=raggedright}
\captionsetup[table]   {labelfont=bf,textfont=it,format=hang,justification=raggedright}
\captionsetup[figure]  {labelfont=bf,textfont=it,format=hang,justification=raggedright}

\captionsetup[figure]  {name=Abbildung} 		% Abbildungen
\captionsetup[table]   {name=Tabelle} 			% Tabellen
\captionsetup[longlisting]{name=Codeabschnitt}	% Codeabschnitte -> fkt. leider nicht


% kapitelweise Nummerierung
\numberwithin{figure}{section}				% kapitelweise Nummerierung für Grafiken
\numberwithin{table}{section}				% kapitelweise Nummerierung für Tabellen
\numberwithin{listing}{section}				% kapitelweise Nummerierung für Codeabschnitte

\usepackage{tcolorbox}
\newtcolorbox[blend into=figures]{myfigure}[2][]{float=htb,capture=hbox,
	blend before title code={\fbox{##1}\ },title={#2},every float=\centering,#1}

